\chapter{2a2-pj2ag5製作心得}
%\renewcommand{\baselinestretch}{10.0} %設定行距
\section{李凱新心得}
這次四人的小組很感謝隊友們的強勁的能力,我們很快就有成果,會變色的計分板還有機械式計分板,我在隊伍裡面擔任統整全部人的結果,長時間都是在查程式是什麼意思,學習到很多東西,還學習到onshape 更靈活應用,所以我很感謝隊友們。
\section{王翔楷心得}
這次的專案與上一次最大的不同是人數變多,分工也相對變得更細、更明確,用在前一個專案所學的技巧來對組內做貢獻,這次的協同也讓我更加熟悉了zmq的應用。
\section{李學淵心得}
這次pj2讓我學會了zmqRemoteAPI控BubbleRob並編寫了計時器的程式,對於lua的陣列的產生與取出有更深的了解,以及python與lua的差別有進一步的認識。
\section{張昱棠心得}
透過這次的專案,我深刻了解到如何規劃四人的協同專案,按時完成老師交付的進度,也讓我對onshape的使用更加精進。
\newpage