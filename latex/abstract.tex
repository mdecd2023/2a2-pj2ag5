\renewcommand{\baselinestretch}{1.5} %設定行距
\pagenumbering{roman} %設定頁數為羅馬數字
\clearpage  %設定頁數開始編譯
\sectionef
\addcontentsline{toc}{chapter}{摘~~~要} %將摘要加入目錄
\begin{center}
\LARGE\textbf{摘~~要}\\
\end{center}
\begin{flushleft}
\fontsize{14pt}{20pt}\sectionef\hspace{12pt}\quad 由於矩陣計算、自動求導技術、開源開發環境、多核GPU運算硬體等這四大發展趨勢,促使AI領域快速發展,藉由這樣的契機,將實體機電系統透過虛擬化訓練提高訓練效率,再將訓練完的模型應用到實體上。\\[12pt]

\fontsize{14pt}{20pt}\sectionef\hspace{12pt}\quad 此專案是w3作業所做的泡泡機器人的延伸,繪製機器人後導入CoppeliaSim模擬環境並給予對應設置,使用zmqRemoteAPI與同組組員協同控制bubbleRob,在我們所建立的場景內踢球競賽,並同時加入記分板顯示場上比分狀態。\\

\end{flushleft}
%=--------------------Abstract----------------------=%
\renewcommand{\baselinestretch}{1.5} %設定行距
\addcontentsline{toc}{chapter}{Abstract} %將摘要加入目錄
\begin{center}
\LARGE\textbf\sectionef{Abstract}\\
\begin{flushleft}
\fontsize{14pt}{16pt}\sectionef\hspace{12pt}\quad Due to the four major development trends of multidimensional arrays  computing, automatic differentiation, open source development environment, and multi-core GPUs computing hardware. The rapid development of the AI field has been promoted. In view of this development, the physical mechatronic systems can gain machine learning efficiency through their simulated virtual system training process. And afterwards to apply the trained model into real mechatronic systems.\\[12pt]

\fontsize{14pt}{16pt}\sectionef\hspace{12pt}\quad This project is an extension of the bubble robot created for the w3 assignment. After designing the robot, we integrated it into the CoppeliaSim simulation environment and configured the necessary settings. We used zmqRemoteAPI to collaboratively control the bubbleRob with our team members. In the scene we created, we had a soccer competition where the robots played against each other, and we also added a scoreboard to display the current score on the field.\\
\end{flushleft}